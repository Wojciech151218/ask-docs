\documentclass{article}
\usepackage[utf8]{inputenc}
\usepackage[T1]{fontenc}
\usepackage[polish]{babel}
\usepackage{graphicx}
\usepackage{booktabs}
\usepackage{float}

\title{Dokumentacja Projektu}
\author{Your Name}
\date{\today}

\begin{document}

\maketitle

\section{Wprowadzenie i specyfikacja wymagań}
\subsection{Opis pomysłu rozwiązania systemu komputerowego}
% Tutaj opisz pomysł rozwiązania systemu komputerowego

\subsection{Opis pomysłu realizacji urządzenia}
% Tutaj opisz funkcje urządzenia

\section{Rozwiązania techniczne i opis konstrukcji}
\subsection{Rysunki techniczne}
% Tutaj umieść rysunki techniczne lub zdjęcia komputera i urządzenia z opisem

\section{Schematy i opis}
\subsection{Schemat ideowy}
% Tutaj umieść schemat ideowy z opisem

\subsection{Schematy montażowe}
% Tutaj umieść schematy montażowe lub zdjęcia z opisem:
% - komputer
% - mechanika
% - silniki
% - zasilanie
% - czujniki

\subsection{Rysunek płytki do trawienia}
% Tutaj umieść rysunek płytki do trawienia z opisem procesu

\section{Oprogramowanie}
\subsection{Wykorzystane zasoby mikrokontrolera}
% Opisz wykorzystane zasoby mikrokontrolera

\subsection{Kod źródłowy}
% Tutaj umieść najistotniejsze fragmenty oprogramowania z komentarzami:
% - algorytm
% - obsługa przerwań
% - obsługa czujników
% - obsługa elementów wykonawczych

\section{Kosztorys}
\begin{table}[H]
\centering
\begin{tabular}{lccc}
\toprule
Część & Ilość sztuk & Miejsce zakupu & Cena \\
\midrule
% Tutaj dodaj elementy kosztorysu
\bottomrule
\end{tabular}
\caption{Kosztorys projektu}
\end{table}

\end{document} 